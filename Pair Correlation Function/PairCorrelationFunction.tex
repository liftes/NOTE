\documentclass[12pt, a4paper, oneside, UTF8]{ctexbook}
\usepackage{amsmath, amsthm, amssymb, bm, graphicx, hyperref, mathrsfs, tikz}
\usepackage{subfiles}
\usepackage{geometry} % 页边距
\geometry{a4paper,scale=0.75}
\usepackage{ulem} % 下划线控制
\usepackage{fancyhdr} % 页面页眉页脚控制
\usepackage{xcolor} % 颜色控制
\usepackage{cite} % 导入引用的包,能够使用\cite
\newcommand{\upcite}[1]{\textsuperscript{\textsuperscript{\cite{#1}}}}
\usepackage{gbt7714}
\usepackage{setspace} % 目录控制
% 图注控制
\usepackage{caption}
\captionsetup{labelformat=default,labelsep=space} %去除图表冒号

\hypersetup{ % 超链接颜色控制
colorlinks=true,
linkcolor=black,
citecolor=black,
anchorcolor=gray,
urlcolor=gray,
}

% 罗马字符
\makeatletter
\newcommand{\rmnum}[1]{\romannumeral #1}
\newcommand{\Rmnum}[1]{\expandafter\@slowromancap\romannumeral #1@}
\makeatother

\title{{\Huge{\textbf{pair correlation function}}}}
\author{赵胜达}
\date{2022 年 3 月}
% \linespread{1.5}
\newtheorem{theorem}{定理}[section]
\newtheorem{definition}[theorem]{定义}
\newtheorem{lemma}[theorem]{引理}
\newtheorem{corollary}[theorem]{推论}
\newtheorem{example}[theorem]{例}
\newtheorem{proposition}[theorem]{命题}

\pagestyle{fancy} % 页眉页脚控制
\lhead{}
\chead{}

\setlength{\baselineskip}{1.7em} % 行间距
\setlength{\parindent}{2em}
\setlength{\parskip}{1em}
\setlength{\headheight}{16pt}

\ctexset{ % 各级别标题设置
    section = {
    format={\flushleft \sffamily \heiti \zihao {3}},
    beforeskip={24pt},afterskip={6pt},},
    subsection = {
    format={\flushleft\sffamily\heiti\zihao{4}},
    beforeskip={12pt},afterskip={6pt},}
}

% 目录控制
% \titlecontents{chapter}[4em]{\bfseries \zihao{4}}{\contentslabel{4em}}{\hspace*{-4em}}{~\titlerule*[0.6pc]{$.$}~\contentspage}
% \titlecontents{section}[4em]{\bfseries \zihao{-4}}{\contentslabel{4em}}{\hspace*{-4em}}{~\titlerule*[0.6pc]{$.$}~\contentspage}

% 颜色控制
\definecolor{lightgrayblue}{rgb}{0.30,0.30,0.50}
\definecolor{darkgreen}{RGB}{0,182,150}

% 图标编号
\renewcommand {\thetable} {\thechapter{}.\arabic{table}}
\renewcommand {\thefigure} {\thechapter{}.\arabic{figure}}

% =======================================================================
\begin{document}
% \captionsetup{labelformat=default,labelsep=space} %去除图表冒号

\maketitle

\pagenumbering{roman}
\setcounter{page}{1}
\newpage
\pagenumbering{Roman}
\setcounter{page}{1}
\begin{spacing}{0.5}
    \tableofcontents
\end{spacing}
\newpage
\setcounter{page}{1}
\pagenumbering{arabic}

\chapter{对关联函数与结构因子}
\par{
    聚合物的大小可以通过各种散射实验(光散射,小角度X射线散射和中子散射等)来测量。假设聚合物由在任意$R$处的一系列散射单元组成,这些单元具有散射振幅$a$。且散射向量$k\equiv k_f-k_i$($k_i$和$k_f$是入射和散射光束的波矢量)处的散射强度写为:
    \begin{align}
        \sum_{n, m=1}^{N_{0}} a_{n} a^{*}_{m} \exp \left[\mathrm{i} \boldsymbol{k} \cdot\left(\boldsymbol{R}_{n}-\boldsymbol{R}_{m}\right)\right]
    \end{align}
    其中$N_0$表示系统的全部散射单元数。特别的,对于均质的聚合物,其各单元的散射强度相同,即上式可写为:
    \begin{align}
        |a|^{2} \sum_{n, m=1}^{N_{0}} \exp \left[i \bm{k} \cdot\left(\bm{R}_{n}-\bm{R}_{m}\right)\right]
    \end{align}
    由此我们可以定义{\color{darkgreen}结构因子$g(\bm{k})$}为与系统大小(即$N_0$)无关的变量,即表示无限大系统中单个散射单元对散射的贡献。具体表示为:
    \begin{align}
        g(\boldsymbol{k}) \equiv \frac{1}{N_{0}} \sum_{n, m}^{N_{0}}\left\langle\exp \left[\mathrm{i} \boldsymbol{k} \cdot\left(\boldsymbol{R}_{n}-\boldsymbol{R}_{m}\right)\right]\right\rangle
        \label{g-defined}
    \end{align}

    类似的,我们可以讨论密度函数和结构因子的关系。对$N$个粒子组成的系统,可以定义密度函数如下:
    \begin{align}
        \rho(\bm{r}) = \sum_{n=1}^N \delta^3(\bm{r}-\bm{r}_n)
    \end{align}
    而考虑两个密度场之间的相互作用,则有:
    \begin{align}
        \begin{split}
            \frac{1}{N_0}\int {\rm d}^{3} \bm{x} \rho(\bm{x})\rho(\bm{x}+\bm{r})
            &=\frac{1}{N_0} \sum_{m,n=1}^{N} \int {\rm d}^{3} \bm{x} \delta^{3}\left(\bm{x}-\bm{r}_{n}\right) \delta^{3}\left(\bm{x}-\bm{r}_{m}+\bm{r}\right)\\
            &=\frac{1}{N} \sum_{m,n=1}^{N} \delta^{3}\left(\bm{r}-\bm{r}_{m}+\bm{r}{r}_{n}\right)\\
            &=g(\bm{r})+\delta^{3}(\bm{r})
        \end{split}
    \end{align}
    其中$\delta^{3}(\bm{r})$表示当$m=n$时的附加影响,既有$\left\langle \hat{\rho}(\bm{r}) \hat{\rho}\left(\bm{r}^{\prime}\right) \right\rangle =g(\bm{r})$。

    考虑到$\left\langle \hat{\rho}(\bm{r}) \hat{\rho}\left(\bm{r}^{\prime}\right) \right\rangle$作为配分函数的二阶矩,其在计算自洽场附近的高斯涨落时需要使用,因此需要计算各类理想链的结构因子。
}

\chapter{各类聚合物模型的结构因子}
\section{理想链的结构因子}
\par{
    在理想模型中,我们暂时不考虑聚合物之间的影响,即考虑无限稀溶液。并引入聚合物段数$N$以替代$N_0$,引入公式\eqref{g-defined}即可得到\upcite{Doi1986TheTO}:
    \begin{align}
        g(\boldsymbol{k}) \equiv \frac{1}{N} \sum_{n, m}^{N}\left\langle\exp \left[\mathrm{i} \boldsymbol{k} \cdot\left(\boldsymbol{R}_{n}-\boldsymbol{R}_{m}\right)\right]\right\rangle=\frac{1}{N} \sum_{n, m}\left\langle\frac{\sin \left(|\boldsymbol{k}|\left|\boldsymbol{R}_{n}-\boldsymbol{R}_{m}\right|\right)}{|\boldsymbol{k}|\left|\boldsymbol{R}_{n}-\boldsymbol{R}_{m}\right|}\right\rangle
        \label{g-sin}
    \end{align}
    上式中的恒等变换在数学上成立,方法为引入球坐标替换笛卡尔空间坐标系,既有以下等式关系:
    \begin{align}
        \begin{split}
            A(\bm{k})=&\int \mathrm{d} \bm{r} \exp (-\mathrm{i} k \cdot \boldsymbol{r}) \tilde{\psi}(\bm{r})\\
            =&\int_{0}^{\infty} \mathrm{d} r \, r^{2} \int_{0}^{\pi} \mathrm{d} \theta \sin \theta \int_{0}^{2 \pi} \mathrm{d} \phi \exp (-\mathrm{i} k r \cos \theta) \tilde{\psi}(\bm{r}) \\
            =&\int_{0}^{\infty} \mathrm{d} r 4 \pi r^{2} \frac{\sin k r}{k r} \tilde{\psi}(\bm{r})\\
            =&\left\langle\frac{\sin k r}{k r}\right\rangle_{\tilde{\psi}}
        \end{split}
    \end{align}
    其中势能函数$\tilde{\psi}(\bm{r})$只与$\bm{r}$的大小$r$有关,在公式\eqref{g-sin}中有$\tilde{\psi}(\bm{r})=1$,而还有:
    \begin{align}
        \langle\ldots\rangle_{\tilde{\psi}}=\int_{0}^{\infty} \mathrm{d} r 4 \pi r^{2} \tilde{\psi}(r) \ldots
    \end{align}
    
    而通常我们关心的有效散射都集中在小波区间内,即$k \rightarrow 0$。因此可以继续对上式进行展开,得到:
    \begin{align}
        \begin{split}
            g(k) &=\frac{1}{N} \sum_{n, m}\left\langle 1-\frac{1}{6} k^{2}\left(R_{n}-R_{m}\right)^{2}+\ldots\right\rangle \\
            &=N\left(1-\frac{1}{3} k^{2} R_{g}^{2}+\ldots\right)
        \end{split}
    \end{align}
    其中$R_g$为均方回转半径,且有$R_{g}^{2} \equiv \frac{1}{2 N^{2}} \sum_{n, m=1}^{N}\left\langle\left(\boldsymbol{R}_{n}-\boldsymbol{R}_{m}\right)^{2}\right\rangle$。
}

\section{高斯链的结构因子}
\par{
    而对于线性高斯链,我们可求其回转半径:
    \begin{align}
        \begin{split}
            R_{g}^{2} & \equiv \frac{1}{2 N^{2}} \sum_{n, m=1}^{N}\left\langle\left(\boldsymbol{R}_{n}-\boldsymbol{R}_{m}\right)^{2}\right\rangle \\
            &=\frac{1}{2 N^{2}} \sum_{n=1}^{N} \sum_{m=1}^{N}|n-m| b^{2}=\frac{1}{2 N^{2}} \int_{0}^{N} \mathrm{d} n \int_{0}^{N} \mathrm{d} m|n-m| b^{2} \\ 
            &=\frac{1}{N^{2}} \int_{0}^{N} \mathrm{d} n \int_{0}^{n} \mathrm{d} m(n-m) b^{2}=\frac{1}{6} N b^{2} 
        \end{split}
    \end{align}
    数学上还有\upcite{Doi1986TheTO}${\rm ^{2.\uppercase\expandafter{\romannumeral1}.20}}$:
    \begin{align}
        \begin{split}
            \left\langle\exp \left[\mathrm{i} \bm{k} \cdot\left(\boldsymbol{R}_{n}-\boldsymbol{R}_{m}\right)\right]\right\rangle &=\left\langle\exp \left[\sum_{\alpha=x, y, z} \mathrm{i} k_{\alpha}\left(R_{n \alpha}-R_{m \alpha}\right)\right]\right\rangle \\
            &=\exp \left[-\frac{1}{2} \sum_{\alpha=x, y, z} k_{\alpha}^{2}\left\langle\left(R_{m \alpha}-R_{n \alpha}\right)^{2}\right\rangle\right]\\
            &=\exp \left[-\frac{b^{2} \boldsymbol{k}^{2}}{6}|n-m|\right]
            \label{g-Gaussian-part}
        \end{split}
    \end{align}
    其中有$\left\langle\left(R_{m \alpha}-R_{n \alpha}\right)^{2}\right\rangle=|m-n| b^{2}/3$,将公式\eqref{g-Gaussian-part}代入定义式\eqref{g-defined}进而得到:
    \begin{gather}
        g(k)=\frac{1}{N} \int_{0}^{N} \mathrm{d} n \int_{0}^{N} \mathrm{d} m \exp \left[-\frac{b^{2} k^{2}}{6}|n-m|\right]=N f\left(k^{2} R_{g}^{2}\right)
        \label{g-Gaussian} \\
        f(x)=\frac{2}{x^{2}}\left(\mathrm{e}^{-x}-1+x\right)
    \end{gather}
    \footnotesize{\color{lightgrayblue}
    附:上式的详细推导如下。首先定义有:
    \begin{align*}
        \frac{b^2}{6} k^2 = \frac{R_g^2}{N} k^2 \equiv \frac{x}{N}
    \end{align*}
    一般地,数学上可计算积分:
    \begin{gather*}
        \int \mathrm{d}m \mathrm{e}^{-xm} = -\frac{1}{x} \mathrm{e}^{-xm}\\
        \int \mathrm{d}m \mathrm{e}^{xm} = \frac{1}{x} \mathrm{e}^{xm}
    \end{gather*}
    代入公式\eqref{g-Gaussian}有:
    \begin{align*}
        g(k)=&\frac{1}{N} \int_{0}^{N} \mathrm{d} n \int_{0}^{N} \mathrm{d} m \exp \left[-\frac{b^{2} k^{2}}{6}|n-m|\right]\\
        =&\frac{1}{N} \int_{0}^{N} \mathrm{d} n \int_{0}^{N} \mathrm{d} m \exp \left[-\frac{x}{N}|n-m|\right]\\
        \text{(积分上限变换)}=&\frac{N}{x^2} [
            \int_0^n \mathrm{d}m \int_0^x \mathrm{d}n \exp(-n+m) ~\text{if}(n>m)\\
            &+\int_0^m \mathrm{d}n \int_0^x \mathrm{d}m \exp(-m+n) ~\text{if}(n  \leq  m)
        ]\\
        \text{(重新定义哑指标统一两积分)}=&\frac{2N}{x^2}
        \int_0^x \mathrm{d}n \mathrm{~e}^{-n} \int_0^n \mathrm{d}m \mathrm{~e}^{m}\\
        =&\frac{2N}{x^2}
        \int_0^x \mathrm{d}n \mathrm{~e}^{-n} [\mathrm{~e}^n - 1]\\
        =&\frac{2N}{x^2}
        \int_0^x \mathrm{d}n ~(1 - \mathrm{e}^{-n})\\
        =&\frac{2N}{x^2} (x+\mathrm{e}^{-x}-1)
    \end{align*}
    (将求和表达式按照爱因斯坦求和约定写出来之后,用来求和的指标,叫做哑指标。)
    }
}

\section{二嵌段聚合物}
\par{
    假设聚合物仍然符合线性高斯链模型,区别在于聚合物整体由A、B两种不同结构的片段单元依次连接而成。设聚合物总片段数为$N$,其中A组分有连续的$n_A$段,B组分有连续的$n_B$段,显然有$N=n_A+n_B$。并定义组分比$f=n_A/N$。

    由于依旧采用高斯链的基本模型,因此上一节中关于均方根半径的推导依然成立。即公式\eqref{g-Gaussian}依然适用,通过调整积分区间即可计算A、B两不同组分结构间的结构因子为:
    \begin{align}
        \label{g-AB}
        \begin{split}
            g(k)=&\frac{1}{N} \int_{0}^{Nf} \mathrm{d} n \int_{Nf}^{N} \mathrm{d} m \exp \left[-\frac{b^{2} k^{2}}{6}|n-m|\right] \\
            =&\frac{1}{N} \int_{0}^{Nf} \mathrm{d}n \int_{Nf}^{N} \mathrm{d}m \exp \left[\frac{x}{N}(n-m)\right] ~(n<m)\\
            =&\frac{N}{x^2} \int_{0}^{xf} \mathrm{d}n \int_{xf}^{x} \mathrm{d}m \exp \left[n-m\right]\\
            =&\frac{N}{x^2} \int_{0}^{xf} \mathrm{d}n ~\mathrm{e}^n
            \int_{xf}^{x} \mathrm{d}m ~\mathrm{e}^{-m}\\
            =&\frac{N}{x^2} (\mathrm{e}^{xf} - 1) \times
            (\mathrm{e}^{-xf} - \mathrm{e}^{-x}) \\
            =&\frac{N}{x^2} (\mathrm{e}^{-x} - \mathrm{e}^{-xf}
            - \mathrm{e}^{-x(1-f)} + 1) \\
            =&\frac{N}{x^2} (\mathrm{e}^{-x} - 2\mathrm{e}^{-x/2} +1)
            ~(\text{when}~ f=1/2)\\
            =&NF(x)
        \end{split}
    \end{align}
    \begin{gather}
        F(x) = \frac{1}{x^2} (\mathrm{e}^{-x} - 2\mathrm{e}^{-x/2} +1) \\
        F^{f}(x) = \frac{1}{x^2} (\mathrm{e}^{-x} - \mathrm{e}^{-xf}
        - \mathrm{e}^{-x(1-f)} + 1)
    \end{gather}

    而考虑二嵌段聚合物整体的结构因子时,应有:
    \begin{gather}
        \label{C-factors}
        C(x) = C_{AA}(x)+C_{BB}(x)+C_{AB}(x)+C_{BA}(x)\\
        C_{AA}(x) = g_D(x)\\
        C_{BB}(x) = g_D(x)\\
        C_{AB}(x) = C_{BA}(x) = F(x)
    \end{gather}
    其中有$g_D(x)=2(xf+\mathrm{e}^{-xf}-1)/x^2$为一般高斯链的结构因子。

    \footnotesize{\color{lightgrayblue}
    附注:结构因子的推导与均方回转半径$R_g$的定义严格相关,一般地,均选取整条高分子链的均方回转半径代入公式中进行推导计算。如理想单高斯链便有$\frac{b^2}{6} k^2 = \frac{R_g^2}{N} k^2 \equiv \frac{x}{N}$。特别地,在文献中\upcite{ZhangChen-14},为统一星型聚合物和游离在溶液中的离散的短聚合物支链,我们定义每段相同组分的聚合物的片段长度为$N$,因此均方回转半径的计算也与此关联。在此条件下需要对公式\eqref{g-AB}中积分的上下限做出对应变换。既有:
    \begin{align}
        \begin{split}
            g(k)=&\frac{1}{N} \int_{0}^{N} \mathrm{d} n \int_{N}^{2N} \mathrm{d} m \exp \left[-\frac{b^{2} k^{2}}{6}|n-m|\right] \\
            =&\frac{1}{N} \int_{0}^{N} \mathrm{d}n \int_{N}^{2N} \mathrm{d}m \exp \left[\frac{x}{N}(n-m)\right] ~(n<m)\\
            =&\frac{N}{x^2} \int_{0}^{x} \mathrm{d}n \int_{x}^{2x} \mathrm{d}m \exp \left[n-m\right]\\
            =&\frac{N}{x^2} \int_{0}^{x} \mathrm{d}n ~\mathrm{e}^n
            \int_{x}^{2x} \mathrm{d}m ~\mathrm{e}^{-m}\\
            =&\frac{N}{x^2} (\mathrm{e}^{x} - 1) \times
            (\mathrm{e}^{-x} - \mathrm{e}^{-2x}) \\
            =&\frac{N}{x^2} (\mathrm{e}^{-2x} - 2\mathrm{e}^{-x} +1)\\
            =&NF(x)
        \end{split}
    \end{align}
    }
}

\section{支链滑动高分子}
\par{
    \begin{figure}[ht]
        \centering
        \includegraphics[scale=1]{figure/split.png}
        \caption{滑动聚合物示意图}
        \label{Figure-SplitPloyer}
    \end{figure}
    一般地,对于含且只含有A、B两种组分的高分子体系,公式\eqref{C-factors}恒成立。如图\ref{Figure-SplitPloyer},现在考虑高分子系统由A、B两种组分构成,且有一A组分支链可在A链条上任意滑动。其中A组分主链和支链的片段长度相等且等于B组分主链长度,既有$N_{A1}=N_{A2}=N_{B}=N/3$。现在需要考虑该高分子链的结构因子。可知分别求解$C_{AA}(x)$,$C_{BB}(x)$,$C_{AB}(x)$三个函数即可。

    不妨定义$f \equiv \kappa / N_{A1}$,且$f$先取固定值,最后对其做在$[0,1]$区间上的积分。不同组分浓度可变后整条链的结构因子可以按照等式$C(x)=C_{AA}(x)+2C_{AB}(x)+C_{BB}(x)$进行展开。其中每一项又由可变链长的单链结构因子$g_D(x,n)$和可变链长的二嵌段结构因子$F(x,m,n)$组成,其中$m,n$表示对应支链的片段长度。

    由于$f_{A}$与$f_{A2}$的定义并不独立,为方便推导可以重新定义A组分浓度$\phi_A = f_A $,滑动支链浓度$\phi_a = N_{A2}/(N_{A1}+N_{A2})$。即保证了$\phi_A$与$\phi_a$的独立性。

    一般地,依旧取整条链的片段数$N$作为推导标准,即定义有$x=k^2R_g^2$,其中$R_g=Nb^2/6=(N_{A1}+N_{A2}+N_{B})b^2/6$为整条链的均方回转半径。则可进行以下计算:
    \begin{align}
        \begin{split}
            g_D(x,n) = \frac{g(x)}{N} =
            &\frac{1}{N^2} \int_{0}^{n} \mathrm{d} p \int_{0}^{n} \mathrm{d} q \exp \left[-\frac{x}{N}|p-q|\right]\\
            =&\frac{2}{x^2}
            \int_0^{nx/N} \mathrm{d}p \mathrm{~e}^{-p} \int_0^{p} \mathrm{d}q \mathrm{~e}^{q}\\
            =&\frac{2}{x^{2}}\left[\exp{(-\frac{n}{N})x}-1+\frac{n}{N}x\right]
        \end{split}\\
        \begin{split}
            F(x,m,n) = \frac{g(x)}{N} = 
            &\frac{1}{N^2} \int_{0}^{m} \mathrm{d}p \int_{m}^{n+m} \mathrm{d}q \exp \left[\frac{x}{N}(p-q)\right] ~(p<q)\\
            = &\frac{1}{x^2} \left[\exp{(-\frac{m+n}{N}x)} 
            - \exp{(-\frac{m}{N}x)}
            - \exp{(-\frac{n}{N})} + 1\right]
        \end{split}
    \end{align}

    同样可先假设滑动支链固定在点$\kappa$处,并有$f \equiv \kappa/N_{A1} = \kappa / [N\phi_A(1-\phi_a)]$。即可得到对应的不同组分间的结构因子:
    \begin{align}
        \begin{split}
            C_{AA}^{\prime}(x) =& 2F(x,\kappa,N_{A1}-\kappa) 
            + 2F(x,\kappa,N_{A2}) 
            + 2F(x,N_{A1}-\kappa,N_{A2})\\
            &+ g_D(x,N_{A2})
            + g_D(x,\kappa)
            + g_D(x,N_{A1}-\kappa)
        \end{split}\\
        \begin{split}
            C_{BB}(x) =& g_D(x,N_{B})
        \end{split}\\
        \begin{split}
            C_{AB}^{\prime}(x) =& F(x,N_{A1},N_{B}) + F(x,N_{A1}+N_{A2}-\kappa,N_{B}) - F(x,N_{A1}-\kappa,N_{B})
        \end{split}
    \end{align}
    其中有:
    \begin{gather*}
        \frac{\kappa}{N} = f\phi_A(1-\phi_a);
        \frac{N_{A1}}{N} = \phi_A(1-\phi_a);
        \frac{N_{A2}}{N} = \phi_A\phi_a ;
        \frac{N_{B}}{N} = 1-\phi_A
    \end{gather*}
}

\section{环形高分子}
\par{
    环形链的结构因子的推导过程不同于线性链,其主要原因在于两点:一是线性链中聚合物段的平移对称性被破坏,取而代之的是环结构中的旋转对称性,二是环形聚合物的均方回转半径$R_g$与线性链不再相同,从而导致在进行积分变量替换时产生额外的常数系数。

    根据参考文献\upcite{Casassa-358},我们可以将环上片段的相互作用视作其所有可能连接路径的线性组合。对应环上任意两点间的连接向量$V_{m,n}$,其受到来自$|m-n|$及$N-|m-n|$两段链段的共同约束。同时已知环形链的均方回转半径$\left\langle R^{2}_g\right\rangle=N b^{2} / 12$。

    首先以线性链的均方回转半径为基准定义变量$\frac{b^2}{6} k^2 = \frac{R_g^2}{N} k^2 \equiv \frac{x}{N}$,进而得到环形链上长为$Nf$的链段的结构因子为:
    \begin{align}
        \begin{split}
            g(k,f)=&\frac{1}{N} \int_{0}^{Nf} \mathrm{d} n \int_{0}^{Nf} \mathrm{d} m \exp \left[-\frac{b^{2} k^{2}}{6}|m-n| \times (N-|m-n|)\right]
        \end{split}
    \end{align}
    通过替换变量可得:
    \begin{align}
        \begin{split}
            g(x,f)=&\frac{1}{N} \int_{0}^{Nf} \mathrm{d} n \int_{0}^{Nf} \mathrm{d} m 
            \exp \left[-\frac{x}{N} |m-n| \times (N-|m-n|)\right]\\
            =&\frac{1}{N}\frac{N}{x} \int_{0}^{\sqrt{xN}f} \mathrm{d} n \int_{0}^{\sqrt{xN}f} \mathrm{d} m 
            \exp \left[-\sqrt{xN} |m-n| + |m-n|^2\right] \\
            &(\sqrt{\frac{x}{N}}m \rightarrow m,
            \sqrt{\frac{x}{N}}n \rightarrow n)\\
            =&\frac{2}{x}\int_{0}^{\sqrt{xN}f} \mathrm{d} m
            \exp{\left[ m^2-\sqrt{xN}m \right]}
            \int_{0}^{m} \mathrm{d} n
            \exp{\left[ \sqrt{xN}n + n^2 - 2mn \right]}\\
            =&\frac{1}{2x} e^{-\frac{p^2}{4}} \left[\sqrt{\pi } (2 f-1) p \left(\text{erfi}\left(\frac{p}{2}\right)-\text{erfi}\left(\frac{p}{2}-f p\right)\right)+2 \left(e^{\frac{p^2}{4}}-e^{\left(f-\frac{1}{2}\right)^2 p^2}\right)\right]
        \end{split}
    \end{align}
    特别地,上式中$p=\sqrt{xN}$(文献中对应的结果为$p=\sqrt{x}$,除此之外结果在形式上完全一致)。而虚误差函数$\text{erfi}(ix)=i \times \text{erf}(x) = \frac{2}{\sqrt{\pi}} \int_{0}^{x} \mathrm{e}^{t^{2}} \mathrm{~d} t $。

    基于此,可以推导环体积分数为$f_r$时的环结构对应的结构因子。只需要改变上式中的积分上下限($Nf \rightarrow Nff_r$)及对应的限制常数($N \rightarrow Nf_r$)即可。既有:
    \begin{align}
        \begin{split}
            g_r(x,f,f_r)=&\frac{1}{N} \int_{0}^{Nff_r} \mathrm{d} n \int_{0}^{Nff_r} \mathrm{d} m 
            \exp \left[-\frac{x}{N} |m-n| \times (Nf_r -|m-n|)\right]\\
            =&\frac{1}{N}\frac{N}{x} \int_{0}^{\sqrt{xN}ff_r} \mathrm{d} n \int_{0}^{\sqrt{xN}ff_r} \mathrm{d} m 
            \exp \left[-\sqrt{xN}f_r |m-n| + |m-n|^2\right] \\
            &(\sqrt{\frac{x}{N}}m \rightarrow m,
            \sqrt{\frac{x}{N}}n \rightarrow n)\\
            =&\frac{2}{x}\int_{0}^{\sqrt{xN}ff_r} \mathrm{d} m
            \exp{\left[ m^2-\sqrt{xN}f_r m \right]}
            \int_{0}^{m} \mathrm{d} n
            \exp{\left[ \sqrt{xN}f_r n + n^2 - 2mn \right]}\\
            =&\frac{1}{2x} e^{-\frac{p^2}{4}} \left[\sqrt{\pi } (2 f-1) p \left(\text{erfi}\left(\frac{p}{2}\right)-\text{erfi}\left(\frac{p}{2}-f p\right)\right)+2 \left(e^{\frac{p^2}{4}}-e^{\left(f-\frac{1}{2}\right)^2 p^2}\right)\right](p=\sqrt{xN}f_r)
        \end{split}
    \end{align}

    类似的,可以考虑环体积分数为$f_r$的环上链段与链体积分数为$f_l$的链上链段间的结构因子为:
    \begin{align}
        \begin{split}
            g_{r-l}(x,f_r,f_l) = & 
            \frac{1}{N}
            \int_0^{Nf_r}\mathrm{d}m 
            \int_{Nf_r}^{N(f_r+f_l)} \mathrm{d}n
            \exp{\left[ \frac{x}{N}[m(Nf_r-m)-n] \right]} \\
            = & \frac{1}{N}
            \int_0^{Nf_r}\mathrm{d}m \exp{\left[ \frac{x}{N}m(Nf_r-m) \right]}
            \int_{Nf_r}^{N(f_r+f_l)} \mathrm{d}n\exp{\left[-\frac{x}{N}n\right]}\\
            = & \frac{1}{N} \frac{N}{x}
            \int_0^{Nf_r}\mathrm{d}m \exp{\left[ \frac{x}{N}m(Nf_r-m) \right]}
            \int_{xf_r}^{x(f_r+f_l)} \mathrm{d}n\exp{\left[-n\right]}
            \\
            = & \frac{(\pi N)^{1/2}}{x^{3/2}}
            \left(e^{f_l x}-1\right) e^{\frac{1}{4} x \left(f_r^2 N-4 (f_l+f_r)\right)}\text{erf}\left(\frac{1}{2} f_r \sqrt{Nx} \right) 
        \end{split}
    \end{align}
    其中误差函数$\operatorname{erf}(x)=\frac{2}{\sqrt{\pi}} \int_{0}^{x} e^{-t^{2}} d t$。

    }

\bibliographystyle{gbt7714-numerical} % 参考文献排版风格
\bibliography{ref} % 导入lib,ref为“ref.lib"的文件名
\end{document}