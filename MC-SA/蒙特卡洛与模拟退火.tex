\documentclass[12pt, a4paper, oneside, UTF8]{ctexbook}
\usepackage{amsmath, amsthm, amssymb, bm, graphicx, hyperref, mathrsfs, tikz, tabularx, float}
\usepackage{booktabs}
\usepackage{subfiles}
\usepackage{geometry} % 页边距
\geometry{a4paper,scale=0.75}
\usepackage{ulem} % 下划线控制
\usepackage{fancyhdr} % 页面页眉页脚控制
\usepackage{xcolor} % 颜色控制
\usepackage{cite} % 导入引用的包,能够使用\cite
\newcommand{\upcite}[1]{\textsuperscript{\textsuperscript{\cite{#1}}}}
\usepackage{gbt7714}
\usepackage{setspace} % 目录控制
% 图注控制
\usepackage{caption}
\captionsetup{labelformat=default,labelsep=space} %去除图表冒号

\hypersetup{ % 超链接颜色控制
colorlinks=true,
linkcolor=black,
citecolor=black,
anchorcolor=gray,
urlcolor=gray,
}

% 罗马字符
\makeatletter
\newcommand{\rmnum}[1]{\romannumeral #1}
\newcommand{\Rmnum}[1]{\expandafter\@slowromancap\romannumeral #1@}
\makeatother

\title{{\Huge{\textbf{蒙特卡洛方法与模拟退火}}}}
\author{赵胜达}
\date{2022 年 6 月}
% \linespread{1.5}
\newtheorem{theorem}{定理}[section]
\newtheorem{definition}[theorem]{定义}
\newtheorem{lemma}[theorem]{引理}
\newtheorem{corollary}[theorem]{推论}
\newtheorem{example}[theorem]{例}
\newtheorem{proposition}[theorem]{命题}

\pagestyle{fancy} % 页眉页脚控制
\lhead{}
\chead{}

\setlength{\baselineskip}{1.7em} % 行间距
\setlength{\parindent}{2em}
\setlength{\parskip}{1em}
\setlength{\headheight}{16pt}

\ctexset{ % 各级别标题设置
    section = {
    format={\flushleft \sffamily \heiti \zihao {3}},
    beforeskip={24pt},afterskip={6pt},},
    subsection = {
    format={\flushleft\sffamily\heiti\zihao{4}},
    beforeskip={12pt},afterskip={6pt},}
}

% 目录控制
% \titlecontents{chapter}[4em]{\bfseries \zihao{4}}{\contentslabel{4em}}{\hspace*{-4em}}{~\titlerule*[0.6pc]{$.$}~\contentspage}
% \titlecontents{section}[4em]{\bfseries \zihao{-4}}{\contentslabel{4em}}{\hspace*{-4em}}{~\titlerule*[0.6pc]{$.$}~\contentspage}

% 颜色控制
\definecolor{lightgrayblue}{rgb}{0.30,0.30,0.50}
\definecolor{darkgreen}{RGB}{0,182,150}

% 图标编号
\renewcommand {\thetable} {\thechapter{}.\arabic{table}}
\renewcommand {\thefigure} {\thechapter{}.\arabic{figure}}

\newcounter{rownumbers}
\newcommand\rownumber{\stepcounter{rownumbers}\arabic{rownumbers}}

% =======================================================================
\begin{document}
% \captionsetup{labelformat=default,labelsep=space} %去除图表冒号

\maketitle

\pagenumbering{roman}
\setcounter{page}{1}
\newpage
\pagenumbering{Roman}
\setcounter{page}{1}
\begin{spacing}{0.5}
    \tableofcontents
\end{spacing}
\newpage
\setcounter{page}{1}
\pagenumbering{arabic}

\chapter{蒙特卡洛方法}
\par{
    蒙特卡罗方法(Monte Carlo,MC)是一种使用随机数(或者更常见的伪随机数)来数值求解计算问题的方法。其基本思想是首先建立一个概率(或随机过程)模型,使它的参数等于问题的解,然后通过对模型(或过程)的抽样来获取有关参数的解的近似值及精度估计\upcite{XJH-80}。

    问题在于,在对热力学系统进行计算机实验时,我们不可能,也没有必要对体系所有的构象态进行遍历,而只需要考虑有代表性的有限的状态进行统计平均即可。因此不能考虑进行全状态空间的简单平均采样,而是需要使用重要性采样算法(important sampling method),如Metropolis算法。

    \section{Metropolis算法}

    \begin{definition}
        细致平衡(detailed balance):\\
        When equilibrium is reached in a reaction system (containing an arbitrary number of components and reaction paths), as many atoms,in their respective molecular.entities will pass forward. as well as backwards.along each individual path in a given finite time interval.Accordingly, the reaction.path in the reverse direction must in every detail be the reverse of the reaction.path in the forward direction (provided always that the system is at equilibrium). The principle of detailed balancing is a consequence for macroscopic systems of the principle of microscopic\upcite{Muller+1994+1077+1184}.
        简言之,细致平衡保证了系统在平衡状态下,不同态之间的跃迁应该满足时间反演对称性,即两者的跃迁几率应该相等。\\        
        而在随机过程理论中,细致平衡条件关系到一个随机过程的可逆性。这和动力学过程或者信息的熵产生,生成元和转移半群的谱性质都有很重要的联系。在计算机模拟的时候,细致平衡条件保证了用势能$U(x)$的梯度构造的马尔科夫链蒙特卡洛过程(MCMC)的不变分布是Boltzmann分布。
    \end{definition}

    在有且仅有一个稳态分布的条件下,\href{https://www.zhihu.com/question/63305712/answer/1804780073}{Matroplis-Hastings 算法}可以产生满足细致平衡的Markov链。对于可逆Markov过程有:
    \begin{align}
        \pi\left(s^{\prime}\right) P\left(s^{\prime} \mid s\right)=\pi(s) P\left(s \mid s^{\prime}\right)
    \end{align}
    因此,对于给定的任意提议分布$q(j|i)$,我们可以通过\href{https://blog.csdn.net/lin360580306/article/details/51240398}{ Matroplis 算法}构造其对应的满足细致平稳性的概率分布。具体做法为,对于不满足细致平衡条件的分布$\pi(i) q(j \mid i) \neq \pi(j) q(i \mid j)$,可以定义其对应的接受概率(acceptance probability)$\alpha(j \mid i)=\pi(j) q(i \mid j)$,而后使不等式两侧相乘,即可构造等式:
    \begin{align}
        \pi(i) q(j \mid i) \alpha(j \mid i)=\pi(j) q(i \mid j) \alpha(i \mid j)
        \label{Tran-P}
    \end{align}
    其中$q(j \mid i) \alpha(j \mid i)$视为新的转移概率。

    但是Metropolis算法构造出的接受概率可能会很小,这样造成算法要经过很多的迭代才能到达平稳分布。为了满足细致平稳,不等式的两边都乘以了一个小的接受概率,那我们可以把其中一个接受概率乘以一个数变为1,另外一边的接受概率也乘上相同的倍数,即可得到 Metropolis-Hastings 算法。既有:
    \begin{gather}
        \pi(i) q(j \mid i) \alpha(j \mid i)=\pi(j) q(i \mid j) \alpha(i \mid j) \nonumber \\
        \pi(i) q(j \mid i) \pi(j) q(i \mid j)=\pi(j) q(i \mid j) \pi(i) q(j \mid i) \\
        \pi(i) q(j \mid i) \frac{\pi(j) q(i \mid j)}{\pi(i) q(j \mid i)}=\pi(j) q(i \mid j) \quad
        \text { or } \quad
        \pi(i) q(j \mid i)=\pi(j) q(i \mid j) \frac{\pi(i) q(j \mid i)}{\pi(j) q(i \mid j)} \nonumber
    \end{gather}
    对比公式\eqref{Tran-P},可以得到:
    \begin{align}
        \alpha(j \mid i)=\min \left\{\frac{\pi(j) q(i \mid j)}{\pi(i) q(j \mid i)}, 1\right\}
        \label{accept-P}
    \end{align}

    因此具体算法流程为:    
    \begin{table}[H]
        \centering
        \begin{tabularx}{\textwidth}{lXXX r}
            \toprule
            \rownumber. & 当前的状态$i$;\\
            \rownumber. & 从提议分布$q(j|i)$中产生新的状态$j$;\\
            \rownumber. & 根据公式\eqref{accept-P}计算接受概率;\\
            \rownumber. & 从$[0,1]$区间中抽样出一个数据点(均匀分布)$u \thicksim U[0,1]$;\\
            \rownumber. & 将随机生成的数和接受率对比,如果$u<\alpha$,状态就从$i$跳转到$j$,否则继续保留在状态$i$。
            \\\bottomrule
        \end{tabularx}%
        \label{tab:01}%
        \caption{Metropolis-Hastings 算法流程}
    \end{table}%

    结合上述算法,可以综合得到高分子的 Metropolis Monte Carlo 算法\upcite{HZW-70,XJH-80,SYT-76,FY-71}:
    \begin{table}[H]
        \centering
        \setcounter{rownumbers}{0}
        \begin{tabularx}{\textwidth}{lXXX r}
            \toprule
            \rownumber. & 生成一个初始构象,并把它作为当前构象;\\
            \rownumber. & 从当前构象出发演化出一个新构象;\\
            \rownumber. & 计算构象改变前后能量的改变;\\
            \rownumber. & 如果$\Delta E<0$,则接受新构象,把它作为当前构象,否则需要计算$\exp{(-\beta \Delta E)}$;\\
            \rownumber. & 从$[0,1]$均匀分布的随机数序列中产生一个随机数$r$,如果$r<\exp{(-\beta \Delta E)}$,接受新构象,并把它作为当前构象;否则保留原构象为当前构象。\\
            \rownumber. & 重复步骤(2)直到$N$足够大,可计算物理量的值:$\left\langle A\left(r^{N}\right)\right\rangle=\frac{1}{M} \sum_{i=1}^{M} A\left(r^{N}\right)$。
            \\\bottomrule
        \end{tabularx}%
        \label{tab:02}%
        \caption{Metropolis Monte Carlo 算法}
    \end{table}%

    \section{高分子的MC模拟模型}

    \subsection{格子链模型-链的生成}
    高分子聚合物是由大量的重复单元组成的大分子。然而,高分子物理关注的是不依赖于单体具体化学构造的聚合物统计性质。因此它从远大于单体间距而远小于链长的尺度着手,把由单体连接的链看成由粗粒化的链节连接的链,每个链节代表实际高分子链中一个较长的链段,这就是所谓的链的粗粒化。而在高分子模拟的建模上,也有格子链模型(1attice chain model)与非格子链模型(off-1attice chain model)两类。在MC中,我们一般考虑前者。
    
    而在链的生成方式上,又分无规行走(random walk,RW)(不考虑任何排除体积)、非即回无规行走(nonreversal random walk,NRRW)(考虑上一步的排除体积)和自避行走(self-avoiding walks,SAW)三类(考虑所有的排除体积)。

    自回避行走模型是聚合物链在良溶剂中的最简单模型\upcite{XJH-80}。良溶剂意味着单体间互相排斥,聚合物链发生溶胀(swollen)。在不同浓度下,指数$\mu$略有变化。需要指出的是,无规行走和自回避行走模型都是“无热的”,温度不起作用。

    \subsection{链的松弛}

    \begin{definition}
        松弛过程(relaxation):\\在外力作用下高分子链由原来的构象过渡到与外力相适应的构象的过程,即高分子链由一种平衡态过渡到另一种平衡态的过程,此过程伴有弹性形变。这主要由于高分子链段的热运动而产生。高分子链段之间有内摩擦,弹性形变需要一定的时间才能完成,此过程称为松弛过程,所需的时间称作松弛时间。
    \end{definition}

    除去使用各类随机行走对系统构象进行初始化外,还要考虑链构象的演化过程,其一般通过松弛算法来进行。对于简立方格子模型一般采用以下松弛模式:纽结跳跃(kink jump)运动,曲柄(crankshaft)运动,蛇行(reptation)运动。

    \begin{figure}[ht]
        \centering
        \includegraphics[width=0.8\textwidth]{figure/SongChi.png}
        \caption{聚合物松弛模式示意图}
        \label{Fig-SongChi}
    \end{figure}

    问题在于一般的松弛算法计算效率较低,尤其是当模拟系统中链节的浓度比较
    高时,空位或溶剂越来越少,链节运动的接受几率将大为下降。因此有新的空穴扩散算法和键长涨落模型来改善弛豫过程。如对角线方法(diagonal bond method),即链的键长允许取三个数值1(方格子边长),$\sqrt{2}$(方格子对角线),$\sqrt{3}$(立方体对角线)。这种方法可以更快速有效地使体系松弛,并产生更多的链构象,因此适
    用于高浓度熔体、支化链和接枝链体系\upcite{XJH-80,Shaffer-82}.
}

\chapter{模拟退火}
\par{
    概括来讲,模拟退火方法是在某一初温下利用Metropolis抽样策略在解空间中进行随机搜索,然后伴随着温度的不断下降重复抽样过程,最终得到问题的全局最优解,即在局部优解中概率性地跳出并最终趋于全局最优解\upcite{XJH-80}。

    采用模拟退火方法,一般需要定义四个基本步骤,即初始构型、能产生所有相关组态的链的试图运动(链松弛)、能量(或目标)函数和降低温度的退火计划(annealing schedule)。对于初始构型和链松弛前面已经进行了介绍。对于目标函数,一般设为系统的能量,其中相互作用能以矩阵表示,矩阵元$E_{ij}=\varepsilon_{ij} k_B T_{ref}$。其中$i,j$指体系中的不同格点的占据单元(如溶剂、A组分聚合物段、B组分聚合物段等),$\varepsilon_{ij}$为约化相互作用能,$k_B$是玻尔兹曼常数,$T_{ref}$为参考温度。对于退火计划,我们设$T_j=fT_{j-1}$。其中$T_j$代表退火到第$j$步时的温度,$f$是标度因子。系统的退火过程将一直持续,直到目标函数或温度达到某个预期值($T_F$)为止。

    即我们首先在足够高的起始温度$T_0$上,对随机产生的链构象进行Monte Carlo抽样。在这个足够高的温度上,实行Metropolis计算法则,可使系统在具有较高能量的微观状态间游动,这个前提相当于“熔化”。然后由温度$T_0$,以一系列递降的设置达到预期的低温$T_F$。在每一温度$T_j$上,要有足够的停留,以保证对组态进行足够次数的Metropolis计算法则的迭代改善,直到系统微观状态的概率分布趋向Boltzmann分布,然后才递降到下一个温度$T_{j+1}$上去。原则上,只要计算时间足够长(即降温速度足够慢),模拟退火方法总能发现最低能量状态,但如果降温速度太快,最后的状态可能不是平衡形态。然而,如果降温速度太慢又会造成计算机资源的浪费。因此,我们通过实验不同的降温速度凭经验确定适当的$f$值、每个温度停留的时间(每个模拟退火步包含的Monte Carlo步)以及退火的步数。需要说明的是,这里引入的温度$T$不一定具有实际的物理意义,它只是一个控制参数。
}

\bibliographystyle{gbt7714-numerical} % 参考文献排版风格
\bibliography{ref} % 导入lib,ref为“ref.lib"的文件名
\end{document}